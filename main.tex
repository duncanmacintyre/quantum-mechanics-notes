\documentclass[11pt]{article}
\usepackage{amsmath,amssymb,amsthm,epsf, graphics,enumerate,fancyhdr,marvosym,cancel}
\usepackage[letterpaper, margin=0.9in]{geometry}

% ordered lists have letters instead of numbers
\renewcommand\theenumi{\alph{enumi}}
\renewcommand\labelenumi{(\theenumi)}

% subsections have letters instead of numbers
\renewcommand{\thesubsection}{\thesection (\alph{subsection})}

% use hyperref for links but don't draw the ugly boxes around links
\usepackage[hidelinks]{hyperref}

% command to create underlined link to outside websites
\newcommand{\hrefunderline}[2]{\underline{\href{#1}{#2}}}

% symbols for common sets
\newcommand{\ZZ}{\mathbb{Z}}
\newcommand{\NN}{\mathbb{N}}
\newcommand{\FF}{\mathbb{F}}
\newcommand{\RR}{\mathbb{R}}
\newcommand{\QQ}{\mathbb{Q}}
\newcommand{\CC}{\mathbb{C}}

% use boldface instead of arrows for vectors
%\renewcommand{\vec}{\mathbf}

% order
\newcommand{\Od}[1]{\mathcal{O}{\left(#1\right)}}

% bras and kets
\newcommand{\bra}[1]{\left\langle#1\right|}
\newcommand{\ket}[1]{\left|#1\right\rangle}
\newcommand{\braket}[2]{\left\langle#1|#2\right\rangle}

% quantum operators
\newcommand{\op}[1]{\hat{#1}}

% for congruences
\newcommand{\mmod}[1]{\;(\operatorname{mod} {#1})}

% new theorem style with boldface title and italic content
\newtheoremstyle{step}%            % Name
  {}%                                     % Space above
  {}%                                     % Space below
  {\itshape}%                           % Body font
  {}%                                     % Indent amount
  {\itshape}%                          % Theorem head font
  {:}%                                    % Punctuation after theorem head
  { }%                                    % Space after theorem head, ' ', or \newline
  {}%                                     % Theorem head spec (can be left empty, meaning `normal')

% new theorem style for a gap in the notes
\newtheoremstyle{gap}%            % Name
  {}%                                     % Space above
  {}%                                     % Space below
  {\itshape}%                           % Body font
  {}%                                     % Indent amount
  {\itshape}%                          % Theorem head font
  {!!!}%                                    % Punctuation after theorem head
  {    }%                                    % Space after theorem head, ' ', or \newline
  {}%                                     % Theorem head spec (can be left empty, meaning `normal')

\theoremstyle{theorem}
\newtheorem{claim}{Claim}[section]
\newtheorem*{claim*}{Claim}
\newtheorem{lemma}[claim]{Lemma}
\newtheorem*{lemma*}{Lemma}
\newtheorem{fact}{Fact of Nature}

\theoremstyle{remark}
\newtheorem*{remark}{Remark}
\newtheorem*{notation}{Notation}

\theoremstyle{step}
\newtheorem{step}{Step}[subsection]
\renewcommand{\thestep}{\arabic{step}}

\theoremstyle{gap}
\newtheorem*{gap}{Gap}

\begin{document}


\title{Quantum Mechanics Notes}
\author{Duncan MacIntyre}
\date{\today}
\maketitle
\tableofcontents
\bigskip
\newpage

\section{Approximation methods}

\subsection{Time-independent perturbation theory}
\newcommand{\ks}{\ket{\psi}}
\newcommand{\ksp}[1]{\ket{\psi^{(#1)}}}
\newcommand{\ksn}[1]{\ket{\psi_{#1}}}
\newcommand{\ksnp}[2]{\ket{\psi^{(#2)}_{#1}}}


Time-independent perturbation theory lets us study Hamiltonians of the form \(H = H_0 + \lambda H^\prime\) where \(H_0\) is a Hamiltonian that we can solve exactly and \(H^\prime\) is some change (a ``perturbation'') that we're interested in. \(\lambda\) is a constant scalar; we will come up with an approximation that is good for small enough values of \(\lambda\).

Our goal is to approximate the eigenstates \(\ks\) and eigenvalues \(E\) of the eigenvalue problem \[H \ks = E \ks.\]
One could imagine that there might be some states \(\ksp{m}\) and values \(E^{(m)}\) so that we can write
\begin{align}
\ks &= \ksp{0} + \lambda \ksp{1} + \lambda^2 \ksp{2} + \cdots \label{tipt.eq.assumption-states}\\
E &= E^{(0)} + \lambda E^{(1)} + \lambda^2 E^{(2)} + \cdots \label{tipt.eq.assumption-states}
\end{align}
for small enough \(\lambda\). We will assume that some approximation of this type works and then try to determine what the \(\ksp{m}\) and \(E^{(m)}\) must be.

\begin{gap}
There should be some justification of whether/why the series above converge and an explanation of the theory of asymptotic approximations that allows us to make these approximations even when the series diverge.
\end{gap}

In our derivations, let us suppose that \(H_0\) has eigenstates \(\ket{n}\) and eigenvalues \(E_n^{(0)}\) for \(n=1, 2, 3, \ldots\), that is, \[H_0 \ket{n} = E_n^{(0)} \ket{n}, \; n=1, 2, 3, \ldots.\] Perturbation theory also works for continuous spectra but it will keep our notation simpler to talk about the discrete case.

\subsubsection{The zero-order energies}

As \(\lambda \to 0\), we should have \(E \to E_n^{(0)}\) for some \(n\). Looking at (\ref{tipt.eq.assumption-states}), we must have \(\boxed{E^{(0)} = E_n^{(0)}.}\) This suggests that we give our state an index. From now on, let's write \(\ksn{n}\) instead of \(\ks\), \(E_n\) instead of \(E\), \(\ksnp{n}{m}\) instead of \(\ksp{m}\), and \(E_n^{(m)}\) instead of \(E^{(m)}\). We have
\begin{align}
\ksn{n} &= \ksnp{n}{0} + \lambda \ksnp{n}{1} + \lambda^2 \ksnp{n}{2} + \cdots \label{tipt.eq.assumption-states2}\\
E_n &= E_n^{(0)} + \lambda E_n^{(1)} + \lambda^2 E_n^{(2)} + \cdots \label{tipt.eq.assumption-states2}
\end{align}
(For a degenerate spectrum, there could be multiple possible \(n\)---just choose one. It turns out that we are able to make this one-to-one map between eigenstates of \(H_0\) and eigenstates of \(H\) but there are multiple possible maps because we could choose any of the states with energy \(E_n\).)

\subsubsection{The zero-order states}

Now, we have \(H_0 \ks = E_n^{(0)} \ks\), so \(\ksn{n}\) is an eigenstate of \(H_0\) with eigenvalue \(E_n^{(0)}\), so \(\ksn{n}\) is a superposition of the states that have energy \(E_n^{(0)}\). Let \(D_n\) be the set of indices \(m\) such that \(E_m^{(0)} = E_n^{(0)}\). Then we can write
\[\ksn{n} = \sum_{m \in D_n} c_m \ket{m}\]
for some coefficients \(c_m\). To find the \(c_m\), we turn to the Schr\"odinger equation. We have
\begin{align*}
\left(H_0 + \lambda H^\prime\right) \ket{\psi_n} &= E_n \ket{\psi_n} \\
\left(H_0 + \lambda H^\prime\right) \left[\ksnp{n}{0} + \lambda \ksnp{n}{1} + \Od{\lambda^2}\right]
&= \left(E_n^{(0)} + \lambda E_n^{(1)} + \Od{\lambda^2}\right) \left[\ksnp{n}{0} + \lambda \ksnp{n}{1} + \Od{\lambda^2}\right] \\
H_0 \ksnp{n}{0} + \lambda \left(H^\prime \ksnp{n}{0} + H_0 \ksnp{n}{1}\right) + \Od{\lambda^2}
&= E_n^{(0)} \ksnp{n}{0} + \lambda \left(E_n^{(0)} \ksnp{n}{1} + E_n^{(1)} \ksnp{n}{0}\right) + \Od{\lambda^2}.
\end{align*}
In the zeroth order of \(\lambda\) this gives \(H_0 \ksnp{n}{0} = E_n^{(0)} \ksnp{n}{0}\), confirming what we already know: \(\ksnp{n}{0}\) is an eigenstate of \(H_0\) with eigenvalue \(E_n^{(0)}\). In the first order of \(\lambda\) we get
\[H^\prime \ksnp{n}{0} + H_0 \ksnp{n}{1} = E_n^{(0)} \ksnp{n}{1} + E_n^{(1)} \ksnp{n}{0}.\]
Multiplying with \(\bra{m}\), where \(m \in D_n\),
\begin{align*}
\bra{m} H^\prime \ksnp{n}{0} + \bra{m}H_0\ket{\psi_n^{(1)}} &= {E_n^{(0)} \braket{m}{\psi_n^{(1)}}} + E_n^{(1)} \braket{m}{\psi_n^{(0)}} \\
\bra{m} H^\prime \ksnp{n}{0} + \cancel{E_m^{(0)}\braket{m}{\psi_n^{(1)}}} &= \cancel{E_n^{(0)} \braket{m}{\psi_n^{(1)}}} + E_n^{(1)} \braket{m}{\psi_n^{(0)}} \\
\braket{m}{\psi_n^{(0)}} &= \frac{\bra{m} H^\prime \ksnp{n}{0}}{E_n^{(1)}}.
\end{align*}



\begin{gap}
This part needs to be added. It should include discussion of the orthonormality of the states.
\end{gap}

\subsubsection{Rearranging the Schr\"odinger equation}

To get the first- and second-order corrections, we start by multiplying the Schr\"odinger equation by \(\bra{\psi_m^{(0)}}\). We get
\begin{align}
\bra{\psi_m^{(0)}} (H_0 + \lambda H^\prime) \ksn{n} &= E_n \braket{\psi_m^{(0)}}{\psi_n} \nonumber\\
E_m^{(0)} \braket{\psi_m^{(0)}}{\psi_n} + \bra{\psi_m^{(0)}} \lambda H^\prime \ksn{n} &= E_n \braket{\psi_m^{(0)}}{\psi_{n}} \nonumber\\
\braket{\psi_m^{(0)}}{\psi_n} &= \frac{\bra{\psi_m^{(0)}}\lambda H^\prime \ket{\psi_n}}{E_n - E_m^{(0)}} \nonumber\\
\braket{\psi_m^{(0)}}{\psi_n} &= \frac{\lambda\bra{\psi_m^{(0)}} H^\prime \left(\ksnp{n}{0} + \lambda \ksnp{n}{1} + \Od{\lambda^2}\right)}{E_n - E_m^{(0)}} \label{eq:tdpt.sefrac}
\end{align}
Let's examine the denominator. We'll consider the two cases where \(n\) is/is not degenerate with \(m\).

Let \(D_n\) be the set of indices \(m\) such that \(E_m^{(0)} = E_n^{(0)}\). If \(m \not\in D_n\) then
\begin{align*}
E_n - E_m^{(0)} &= {\left(E_n^{(0)} - E_m^{(0)}\right)} + \lambda E_n^{(1)} + \lambda^2 E_n^{(1)} + \Od{\lambda^3}\\
&= {\left(E_n^{(0)} - E_m^{(0)}\right)}\left[1 + \Od{\lambda}\right].
\end{align*}
Using the expansion \(\frac{1}{1+x} = 1 - x + x^2 - x^3 + \cdots\) with \(x = \Od{\lambda}\),
\[E_n - E_m^{(0)} = \left(E_n^{(0)} - E_m^{(0)}\right) \frac{1}{1 - \Od{\lambda}}\]
and plugging this in to (\ref{eq:tdpt.sefrac}) we get
\begin{align}
\braket{\psi_m^{(0)}}{\psi_n} &= \frac{\lambda \bra{\psi_m^{(0)}} H^\prime \left(\ksnp{n}{0} + \lambda \ksnp{n}{1} + \Od{\lambda^2}\right)}{E_n^{(0)} - E_m^{(0)}}\left[1+\Od{\lambda^2}\right] \nonumber\\
&= \frac{\lambda \bra{\psi_m^{(0)}} H^\prime \ksnp{n}{0}}{E_n^{(0)} - E_m^{(0)}}+ \Od{\lambda^2},
&& n \not\in D_n. \label{eq:tdpt.nDn}
\end{align}

If instead \(m \in D_n\) then
\begin{align*}
E_n - E_m^{(0)} &= \cancel{\left(E_n^{(0)} - E_m^{(0)}\right)} + \lambda E_n^{(1)} + \lambda^2 E_n^{(1)} + \Od{\lambda^3}\nonumber\\
&= \lambda E_n^{(1)} \left(1 + \lambda \left[\frac{E_n^{(2)}}{E_n^{(1)}} + \Od{\lambda} \right]\right)
\end{align*}
Again, we use the expansion \(\frac{1}{1+x} = 1 - x + x^2 - x^3 + \cdots\), but now with \(x = \lambda \left[\frac{E_n^{(2)}}{E_n^{(1)}} + \Od{\lambda}\right]\). We get
\[
E_n - E_m^{(0)} = \left(E_n^{(0)} - E_m^{(0)}\right) + \frac{\lambda E_n^{(1)}}{1 - \lambda \frac{E_n^{(2)}}{E_n^{(1)}} + \Od{\lambda^2}}.
\]
Plugging this in to (\ref{eq:tdpt.sefrac}) we get
\begin{align}
\braket{\psi_m^{(0)}}{\psi_n} &= \frac{\bra{\psi_m^{(0)}} H^\prime \left(\ksnp{n}{0} + \lambda \ksnp{n}{1} + \Od{\lambda^2}\right)}{E_n^{(1)}}\left[1- \lambda \frac{E_n^{(2)}}{E_n^{(1)}} + \Od{\lambda^2} \right] \nonumber\\
&= \frac{\bra{\psi_m^{(0)}} H^\prime \ket{\psi_n^{(0)}}}{E_n^{(1)}} + \lambda \left[\frac{\bra{\psi_m^{(0)}} H^\prime \ket{\psi_n^{(1)}}}{E_n^{(1)}} - \frac{E_n^{(2)} \bra{\psi_m^{(0)}} H^\prime \ket{\psi_n^{(0)}}}{{\left[E_n^{(1)}\right]}^2}\right] + \Od{\lambda^2},
&& n \in D_n. \label{eq:tdpt.Dn}
\end{align}

Now, we can write an expression for complete state. We have \(\ket{\psi_n} = \sum_m \ket{\psi_m^{(0)}} \braket{\psi_m^{(0)}}{\psi_n}\) and we can read off the \(\braket{\psi_m^{(0)}}{\psi_n}\) from (\ref{eq:tdpt.nDn}) and (\ref{eq:tdpt.Dn}). Putting this all together,
\begin{align}
\ksnp{n}{0} + \lambda \ksnp{n}{1} + \Od{\lambda^2} =
&\sum_{m \in D_n}\ket{\psi_m^{(0)}}\frac{\bra{\psi_m^{(0)}} H^\prime \ket{\psi_n^{(0)}}}{E_n^{(1)}}
\\\nonumber&+\lambda \left\{
\sum_{m \in D_n}\ket{\psi_m^{(0)}} \left[\frac{\bra{\psi_m^{(0)}} H^\prime \ket{\psi_n^{(1)}}}{E_n^{(1)}} - \frac{E_n^{(2)} \bra{\psi_m^{(0)}} H^\prime \ket{\psi_n^{(0)}}}{{\left[E_n^{(1)}\right]}^2}\right]\right.
\\\nonumber&\hspace{3em}\left.+ \sum_{m \not\in D_n} \ket{\psi_m^{(0)}}\frac{\bra{\psi_m^{(0)}} H^\prime \ksnp{n}{0}}{E_n^{(0)} - E_m^{(0)}}\right\}
\\\nonumber&+\Od{\lambda^2}. 
\end{align}
Equating orders of \(\lambda\), we have a zero-order equation
\begin{equation}\label{eq:tdpt.zeroordereq}
\boxed{\ksnp{n}{0} = \sum_{m \in D_n}\ket{\psi_m^{(0)}}\frac{\bra{\psi_m^{(0)}} H^\prime \ket{\psi_n^{(0)}}}{E_n^{(1)}}}
\end{equation}
and a first-order equation
\begin{equation}\label{eq:tdpt.firstordereq}
\boxed{\ksnp{n}{1} = \sum_{m \in D_n}\ket{\psi_m^{(0)}}
\left[\frac{\bra{\psi_m^{(0)}} H^\prime \ket{\psi_n^{(1)}}}{E_n^{(1)}} - 
\frac{E_n^{(2)} \bra{\psi_m^{(0)}} H^\prime \ket{\psi_n^{(0)}}}{{\left[E_n^{(1)}\right]}^2}\right]
+ \sum_{m \not\in D_n} \ket{\psi_m^{(0)}}\frac{\bra{\psi_m^{(0)}} H^\prime \ksnp{n}{0}}{E_n^{(0)} - E_m^{(0)}}.}
\end{equation}

\subsubsection{First order energy}

Multiplying (\ref{eq:tdpt.zeroordereq}) by \(\bra{\psi_n^{(0)}}\),
\begin{align*}
\braket{\psi_n^{(0)}}{\psi_n^{(0)}} &= \sum_{m \in D_n}\braket{\psi_n^{(0)}}{\psi_m^{(0)}}\frac{\bra{\psi_m^{(0)}} H^\prime \ket{\psi_n^{(0)}}}{E_n^{(1)}} \\
1 &= \sum_{m \in D_n}\delta_{mn}\frac{\bra{\psi_m^{(0)}} H^\prime \ket{\psi_n^{(0)}}}{E_n^{(1)}}
\end{align*}
so
\[
\boxed{E_n^{(1)} = \bra{\psi_n^{(0)}} H^\prime \ket{\psi_n^{(0)}}}.
\]
In words, the first order energy correction is the expectation value of \(H^\prime\) for the zero-order state.




\section{Identical particles}

Suppose we have a system of two identical particles with the wavefunction \(\Psi(x_1, x_2)\) where \(x_1\) and \(x_2\) are coordinates of the particles (taking into account things like position and spin).

\begin{fact}
Particles either have wavefunctions that are symmetric, with \(\Psi(x_1, x_2) = \Psi(x_2, x_1)\), or antisymmetric, with \(\Psi(x_1, x_2) = -\Psi(x_2, x_1)\).
\end{fact}

We call the symmetric type of particles bosons and the antisymmetric type of particles fermions.

\subsection{Pauli exclusion principle}

Consider a wavefunction \(\Psi(x_1, x_2)\) for two fermions. If \(x_1 = x_2 = x\) then we have \(\Psi(x, x) = - \Psi(x, x)\) so it must be that \(\Psi(x, x) = 0\). This shows that two identical fermions cannot occupy the same state.







\newpage
\section{Time-dependent perturbation theory in the interaction picture}

{\bf Perturbation theory setup.}
Suppose we have a Hamiltonian of the form \[\op{H} = \op{H}_0 + \op{V}(t)\]
where \(\op{H}_0\) is a well-understood Hamiltonian that does not depend on time and \(\op{V}(t)\) is ``small''. For example, \(\op{H}_0\) might be the free particle Hamiltonian \(\op{H}_0 = \frac{m}{2} \nabla^2\). Our equation of motion is the Schr\"odinger equation
\[i \hbar \frac{\partial}{\partial t} \ket{\Psi, t} = \left(\op{H}_0 + \op{V}(t)\right) \ket{\Psi, t}\]
where \(\ket{\Psi, t}\) is the usual Schr\"odinger-picture state at time \(t\).

Let \(\op{U}(t_0, t) = e^{-i\op{H}_0(t-t_0)/\hbar}\). Then \(\op{U}(t_0, t)\) is the operator that evolves a state from time \(t_0\) to time \(t\) according to \(\op{H}_0\). We define the interaction-picture state to be
\[\ket{\Psi_I, t} = \op{U}(t_0, t)^\dagger \ket{\Psi, t}\]
so \(\ket{\Psi, t} = \op{U}(t_0, t) \ket{\Psi_I, t}\).
Plugging this in to the Schr\"odinger equation,
\begin{align*}
i \hbar \frac{\partial}{\partial t} \op{U}(t_0, t)\ket{\Psi_I, t} &= \left(\op{H}_0 + \op{V}(t)\right) \op{U}(t_0, t)\ket{\Psi, t} \\
i \hbar \left[\frac{\partial}{\partial t} e^{-i\op{H}_0(t-t_0)/\hbar}\right] \ket{\Psi_I, t} + i \hbar e^{-i\op{H}_0(t-t_0)/\hbar} \frac{\partial}{\partial t} \ket{\Psi_I, t} &= \op{H}_0 e^{-i\op{H}_0(t-t_0)/\hbar}\ket{\Psi_I, t} + \op{V}(t) e^{-i\op{H}_0(t-t_0)/\hbar}\ket{\Psi_I, t} \\
\cancel{-i^2 \op{H}_0 e^{i\op{H}_0(t-t_0)/\hbar} \ket{\Psi_I, t}} + i \hbar e^{-i\op{H}_0(t-t_0)/\hbar} \frac{\partial}{\partial t} \ket{\Psi_I, t} &= \cancel{\op{H}_0 e^{-i\op{H}_0(t-t_0)/\hbar}\ket{\Psi_I, t}} + \op{V}(t) e^{-i\op{H}_0(t-t_0)/\hbar}\ket{\Psi_I, t} \\
 i \hbar \frac{\partial}{\partial t} \ket{\Psi_I, t} &= e^{i\op{H}_0(t-t_0)/\hbar}\op{V}(t) e^{-i\op{H}_0(t-t_0)/\hbar}\ket{\Psi_I, t} \\
 i \hbar \frac{\partial}{\partial t} \ket{\Psi_I, t} &= \op{H}_I(t)\ket{\Psi_I, t}
\end{align*}
where we define the interaction Hamiltonian to be
\[\op{H}_I(t) = \op{U}(t_0, t)^\dagger \op{V}(t)\op{U}(t_0, t) = e^{i\op{H}_0(t-t_0)/\hbar}\op{V}(t) e^{-i\op{H}_0(t-t_0)/\hbar}.\]

This sets up the interaction picture. We have rephrased our problem so that we can continue with quantum mechanics normally without having to worry about the time evolution due to \(\op{H}_0\).

We now integrate both sides of our expression.
\begin{align}
\int_{t_0}^t \frac{\partial}{\partial t^\prime} \ket{\Psi_I, t^\prime} &= \frac{i}{\hbar} \int_{t_0}^t \op{H}_I(t^\prime)\ket{\Psi_I, t^\prime} \,dt^\prime \nonumber\\
\ket{\Psi_I, t} - \ket{\Psi_I, t_0} &= \frac{i}{\hbar} \int_{t_0}^t \op{H}_I(t^\prime)\ket{\Psi_I, t^\prime} \,dt^\prime \nonumber\\
\ket{\Psi_I, t} &= \ket{\Psi_I, t_0} + \frac{i}{\hbar} \int_{t_0}^t \op{H}_I(t^\prime)\ket{\Psi_I, t^\prime} \,dt^\prime \label{eq.intse}
\end{align}
This is called the integral form of the Schr\"odinger equation.

We can now iteratively calculate perturbative approximations where we assume \(\op{H}_I(t)\) is small. The zero order approximation is simply
\[\ket{\Psi_I, t} = \ket{\Psi_I, t_0} + \Od{\op{H}_I(t)}.\]
Plugging this in for the state inside the integral in (\ref{eq.intse}), we get the first order approximation
\begin{equation}\label{eq.firsttdpe}
\ket{\Psi_I, t} = \ket{\Psi_I, t_0} + \frac{i}{\hbar} \int_{t_0}^t \op{H}_I(t^\prime)\ket{\Psi_I, t_0} \,dt^\prime + \Od{\op{H}_I(t)^2}.
\end{equation}
We can plug in again to get the second order approximation
\[
\ket{\Psi_I, t} = \ket{\Psi_I, t_0} + \frac{i}{\hbar} \int_{t_0}^t \op{H}_I(t^\prime)\left[\ket{\Psi_I, t_0} + \frac{i}{\hbar} \int_{t_0}^t \op{H}_I(t^{\prime\prime})\ket{\Psi_I, t_0} \,dt^{\prime\prime}\right] \,dt^\prime + \Od{\op{H}_I(t)^3}.
\]
In general, we can keep going to achieve higher order approximations. For us, however, the first order approximation (\ref{eq.firsttdpe}) is enough.

{\bf Application to scattering.} We will continue for scattering between two particles. We will assume that the particles are distinguishable so that we don't need to restrict ourselves to symmetric or antisymmetric states. We will also assume that the particles interact in a way that only depends on the distance between them. That is, in the position basis, we can write \[\bra{x_1, x_2}\op{V}(t)\ket{\psi} = \int_{x_1,x_2} \frac{d^3 x_1 \,d^3x_2}{(2\pi)^6} V(x_1 - x_2) \braket{x_1, x_2}{\psi}.\] (Note that this is not time dependent.) Take \(\op{H}_0\) to be the free particle Hamiltonian.

It will be easiest to work in the momentum basis. We will do our calculations for momentum eigenstates---that is, plane waves. Write momentum eigenstates as \(\ket{k_1, k_2}\). In the position basis these have wavefunctions
\[\braket{x_1, x_2}{k_1, k_2} = e^{ik_1\cdot x_1 + ik_2\cdot x_2}.\]
Here \(x_1\) and \(x_2\) are the (3-vector) positions of the first and second particles and \(k_1\) and \(k_2\) are the (3-vector) momenta of the first and second particle. Also, we note that \(\ket{k_1, k_2}\) has energy \(\frac{1}{2m} \left({k_1}^2 + {k_2}^2\right)\) and so
\begin{equation}
\label{eq.timeevonk}
\op{U}(t_0, t) \ket{k_1, k_2} = e^{\frac{i\left(t-t_0\right)\left({k_1}^2 + {k_2}^2\right)}{2m\hbar}} \ket{k_1, k_2}.
\end{equation}

Let \(\widetilde{V}(p)\) be the Fourier transform of \(V(x_1 - x_2)\) so that
\begin{equation}
\label{eq.ftV}
V(x_1 - x_2) = \int_{-\infty}^{\infty} \frac{d^3 p}{(2 \pi)^3} \widetilde{V}(p) e^{ip\cdot(x_1 - x_2)}.
\end{equation}

We now consider (\ref{eq.firsttdpe}) and take the limits \(t_0 \to -\infty\) and \(t \to \infty\) because, in scattering, we measure states long before and long after the scattering occurs. Then, dropping the higher-order terms,
\[
\ket{\Psi_I, \infty} = \ket{\Psi_I, -\infty} + \frac{i}{\hbar} \int_{-\infty}^\infty \op{H}_I(t)\ket{\Psi_I, -\infty} \,dt.
\]
Take the initial state to be \(\ket{\Psi_I, -\infty} = \ket{k_1, k_2}\). Multiplying by the bra \(\bra{k_1^\prime, k_2^\prime}\),
\begin{align}
\braket{k_1^\prime, k_2^\prime}{\Psi_I, \infty} &= \braket{k_1^\prime, k_2^\prime}{k_1, k_2} + \frac{i}{\hbar} \int_{-\infty}^\infty \bra{k_1^\prime, k_2^\prime} \op{H}_I(t)\ket{k_1, k_2} \,dt \nonumber\\
&= \delta^3(k_1^\prime - k_1)\delta^3(k_2^\prime - k_2) + \frac{i}{\hbar} \int_{-\infty}^\infty \bra{k_1^\prime, k_2^\prime} \op{H}_I(t)\ket{k_1, k_2} \,dt. \label{eq.intermed}
\end{align}
This motivates us to compute the matrix element
\[
\bra{k_1^\prime, k_2^\prime} \op{H}_I(t)\ket{k_1, k_2} = \bra{k_1^\prime, k_2^\prime} \op{U}(t_0, t)^\dagger \op{V}(t) \op{U}(t_0, t) \ket{k_1, k_2}.
\]
Plugging in (\ref{eq.timeevonk}),
\begin{align*}
\bra{k_1^\prime, k_2^\prime} \op{H}_I(t)\ket{k_1, k_2} &= \bra{k_1^\prime, k_2^\prime} e^{\frac{i\left(t-t_0\right)\left({k_1^\prime}^2 + {k_2^\prime}^2\right)}{2m\hbar}} \op{V}(t) e^{\frac{-i\left(t-t_0\right)\left({k_1}^2 + {k_2}^2\right)}{2m\hbar}} \ket{k_1, k_2} \\
&= e^{\frac{i\left(t-t_0\right)\left({k_1^\prime}^2 + {k_2^\prime}^2 - {k_1}^2 - {k_2}^2\right)}{2m\hbar}} \bra{k_1^\prime, k_2^\prime} \op{V}(t) \ket{k_1, k_2} \\
&= e^{\frac{i\left(t-t_0\right)\left({k_1^\prime}^2 + {k_2^\prime}^2 - {k_1}^2 - {k_2}^2\right)}{2m\hbar}} \bra{k_1^\prime, k_2^\prime} \left\{\int_{x_1, x_2} \frac{d^3 x_1 d^3 x_2}{(2\pi)^6}\ket{x_1, x_2}\bra{x_1, x_2}\right\}\op{V}(t) \ket{k_1, k_2} \\
&= e^{\frac{i\left(t-t_0\right)\left({k_1^\prime}^2 + {k_2^\prime}^2 - {k_1}^2 - {k_2}^2\right)}{2m\hbar}}  \int_{x_1, x_2} \frac{d^3 x_1 d^3 x_2}{(2\pi)^6}\braket{k_1^\prime, k_2^\prime}{x_1, x_2}\bra{x_1, x_2}\op{V}(t) \ket{k_1, k_2} \\
&= e^{\frac{i\left(t-t_0\right)\left({k_1^\prime}^2 + {k_2^\prime}^2 - {k_1}^2 - {k_2}^2\right)}{2m\hbar}}  \int_{x_1, x_2} \frac{d^3 x_1 d^3 x_2}{(2\pi)^6}e^{-ik_1^\prime\cdot x_1 - ik_2^\prime\cdot x_2}V(x_1 - x_2) e^{ik_1\cdot x_1 + ik_2\cdot x_2} 
\end{align*}
and plugging in (\ref{eq.ftV}),
\begin{align*}
\bra{k_1^\prime, k_2^\prime} \op{H}_I(t)\ket{k_1, k_2} 
&= e^{\frac{i\left(t-t_0\right)\left({k_1^\prime}^2 + {k_2^\prime}^2 - {k_1}^2 - {k_2}^2\right)}{2m\hbar}} \int_{-\infty}^{\infty} \frac{d^3p}{(2\pi)^3}\int_{x_1, x_2} \frac{d^3 x_1 d^3 x_2}{(2\pi)^6}e^{-ik_1^\prime\cdot x_1 - ik_2^\prime\cdot x_2}\widetilde{V}(p)e^{ip\cdot(x_1-x_2)} e^{ik_1\cdot x_1 + ik_2\cdot x_2} \\
&= e^{\frac{i\left(t-t_0\right)\left({k_1^\prime}^2 + {k_2^\prime}^2 - {k_1}^2 - {k_2}^2\right)}{2m\hbar}} \int_{-\infty}^{\infty} \frac{d^3p}{(2\pi)^3}\widetilde{V}(p) \int_{x_1, x_2} \frac{d^3 x_1 d^3 x_2}{(2\pi)^6}e^{i(k_1-k_1^\prime+p)\cdot x_1}e^{i(k_2-k_2^\prime-p)\cdot x_2}\\
&= e^{\frac{i\left(t-t_0\right)\left({k_1^\prime}^2 + {k_2^\prime}^2 - {k_1}^2 - {k_2}^2\right)}{2m\hbar}} \int_{-\infty}^{\infty} \frac{d^3p}{(2\pi)^3}\widetilde{V}(p) \,\delta^3(k_1-k_1^\prime+p)\,\delta^3(k_2-k_2^\prime-p)\\
&= e^{\frac{i\left(t-t_0\right)\left({k_1^\prime}^2 + {k_2^\prime}^2 - {k_1}^2 - {k_2}^2\right)}{2m\hbar}} \widetilde{V}(k_1^\prime-k_1) \;\delta^3(k_1^\prime-k_1-k_2^\prime+k_2).
\end{align*}
Putting this in to (\ref{eq.intermed}) we get
\begin{align}
\braket{k_1^\prime, k_2^\prime}{\Psi_I, \infty} =&\;\; \delta^3(k_1^\prime - k_1)\delta^3(k_2^\prime - k_2) + \frac{i}{\hbar} \int_{-\infty}^\infty dt\;e^{\frac{i\left(t-t_0\right)\left({k_1^\prime}^2 + {k_2^\prime}^2 - {k_1}^2 - {k_2}^2\right)}{2m\hbar}} \widetilde{V}(k_1^\prime-k_1) \;\delta^3(k_1^\prime-k_1-k_2^\prime+k_2) \nonumber\\
=& \;\;\delta^3(k_1^\prime - k_1)\;\delta^3(k_2^\prime - k_2) \nonumber\\&\;\;\;\;+ {i}\, \widetilde{V}(k_1^\prime-k_1) \;\delta^3(k_1^\prime-k_1-k_2^\prime+k_2) \;\delta\left(\frac{1}{{2m}}\left({k_1^\prime}^2 + {k_2^\prime}^2 - {k_1}^2 - {k_2}^2\right)\right).
\end{align}
Interestingly, we see that the delta functions cause energy and momentum to be conserved.





\end{document}