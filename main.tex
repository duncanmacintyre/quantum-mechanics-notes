\documentclass[11pt]{article}
\usepackage{amsmath,amssymb,amsthm,epsf, graphics,enumerate,fancyhdr,marvosym,cancel}
\usepackage[letterpaper, margin=0.9in]{geometry}

% ordered lists have letters instead of numbers
\renewcommand\theenumi{\alph{enumi}}
\renewcommand\labelenumi{(\theenumi)}

% subsections have letters instead of numbers
\renewcommand{\thesubsection}{\thesection (\alph{subsection})}

% use hyperref for links but don't draw the ugly boxes around links
\usepackage[hidelinks]{hyperref}

% command to create underlined link to outside websites
\newcommand{\hrefunderline}[2]{\underline{\href{#1}{#2}}}

% symbols for common sets
\newcommand{\ZZ}{\mathbb{Z}}
\newcommand{\NN}{\mathbb{N}}
\newcommand{\FF}{\mathbb{F}}
\newcommand{\RR}{\mathbb{R}}
\newcommand{\QQ}{\mathbb{Q}}
\newcommand{\CC}{\mathbb{C}}

% use boldface instead of arrows for vectors
%\renewcommand{\vec}{\mathbf}

% order
\newcommand{\Od}[1]{\mathcal{O}{\left(#1\right)}}

% bras and kets
\newcommand{\bra}[1]{\left\langle#1\right|}
\newcommand{\ket}[1]{\left|#1\right\rangle}
\newcommand{\braket}[2]{\left\langle#1|#2\right\rangle}

% quantum operators
\newcommand{\op}[1]{\hat{#1}}

% for congruences
\newcommand{\mmod}[1]{\;(\operatorname{mod} {#1})}

% new theorem style with boldface title and italic content
\newtheoremstyle{step}%            % Name
  {}%                                     % Space above
  {}%                                     % Space below
  {\itshape}%                           % Body font
  {}%                                     % Indent amount
  {\itshape}%                          % Theorem head font
  {:}%                                    % Punctuation after theorem head
  { }%                                    % Space after theorem head, ' ', or \newline
  {}%                                     % Theorem head spec (can be left empty, meaning `normal')

% new theorem style for a gap in the notes
\newtheoremstyle{gap}%            % Name
  {}%                                     % Space above
  {}%                                     % Space below
  {\itshape}%                           % Body font
  {}%                                     % Indent amount
  {\itshape}%                          % Theorem head font
  {!!!}%                                    % Punctuation after theorem head
  {    }%                                    % Space after theorem head, ' ', or \newline
  {}%                                     % Theorem head spec (can be left empty, meaning `normal')

\theoremstyle{theorem}
\newtheorem{claim}{Claim}[section]
\newtheorem*{claim*}{Claim}
\newtheorem{lemma}[claim]{Lemma}
\newtheorem*{lemma*}{Lemma}
\newtheorem{fact}{Fact of Nature}

\theoremstyle{remark}
\newtheorem*{remark}{Remark}
\newtheorem*{notation}{Notation}

\theoremstyle{step}
\newtheorem{step}{Step}[subsection]
\renewcommand{\thestep}{\arabic{step}}

\theoremstyle{gap}
\newtheorem*{gap}{Gap}

\begin{document}


\title{Quantum Mechanics Notes}
\author{Duncan MacIntyre}
\date{\today}
\maketitle
\tableofcontents
\bigskip
\newpage

\section{Approximation methods}

\subsection{Time-independent perturbation theory}
\newcommand{\ks}{\ket{\psi}}
\newcommand{\ksp}[1]{\ket{\psi^{(#1)}}}
\newcommand{\ksn}[1]{\ket{\psi_{#1}}}
\newcommand{\ksnp}[2]{\ket{\psi^{(#2)}_{#1}}}


Time-independent perturbation theory lets us study Hamiltonians of the form \(H = H_0 + \lambda H^\prime\) where \(H_0\) is a Hamiltonian that we can solve exactly and \(H^\prime\) is some change (a ``perturbation'') that we're interested in. \(\lambda\) is a constant scalar; we will come up with an approximation that is good for small enough values of \(\lambda\).

Our goal is to approximate the eigenstates \(\ks\) and eigenvalues \(E\) of the eigenvalue problem \[H \ks = E \ks.\]
One could imagine that there might be some states \(\ksp{m}\) and values \(E^{(m)}\) so that we can write
\begin{align}
\ks &= \ksp{0} + \lambda \ksp{1} + \lambda^2 \ksp{2} + \cdots \label{tipt.eq.assumption-states}\\
E &= E^{(0)} + \lambda E^{(1)} + \lambda^2 E^{(2)} + \cdots \label{tipt.eq.assumption-states}
\end{align}
for small enough \(\lambda\). We will assume that some approximation of this type works and then try to determine what the \(\ksp{m}\) and \(E^{(m)}\) must be.

\begin{gap}
There should be some justification of whether/why the series above converge and an explanation of the theory of asymptotic approximations that allows us to make these approximations even when the series diverge.
\end{gap}

In our derivations, let us suppose that \(H_0\) has eigenstates \(\ket{n}\) and eigenvalues \(E_n^{(0)}\) for \(n=1, 2, 3, \ldots\), that is, \[H_0 \ket{n} = E_n^{(0)} \ket{n}, \; n=1, 2, 3, \ldots.\] Perturbation theory also works for continuous spectra but it will keep our notation simpler to talk about the discrete case.

\subsubsection{The zero-order energies}

As \(\lambda \to 0\), we should have \(E \to E_n^{(0)}\) for some \(n\). Looking at (\ref{tipt.eq.assumption-states}), we must have \(\boxed{E^{(0)} = E_n^{(0)}.}\) This suggests that we give our state an index. From now on, let's write \(\ksn{n}\) instead of \(\ks\), \(E_n\) instead of \(E\), \(\ksnp{n}{m}\) instead of \(\ksp{m}\), and \(E_n^{(m)}\) instead of \(E^{(m)}\). We have
\begin{align}
\ksn{n} &= \ksnp{n}{0} + \lambda \ksnp{n}{1} + \lambda^2 \ksnp{n}{2} + \cdots \label{tipt.eq.assumption-states2}\\
E_n &= E_n^{(0)} + \lambda E_n^{(1)} + \lambda^2 E_n^{(2)} + \cdots \label{tipt.eq.assumption-states2}
\end{align}
(For a degenerate spectrum, there could be multiple possible \(n\)---just choose one. It turns out that we are able to make this one-to-one map between eigenstates of \(H_0\) and eigenstates of \(H\) but there are multiple possible maps because we could choose any of the states with energy \(E_n\).)

\subsubsection{The zero-order states}

Now, we have \(H_0 \ks = E_n^{(0)} \ks\), so \(\ksn{n}\) is an eigenstate of \(H_0\) with eigenvalue \(E_n^{(0)}\), so \(\ksn{n}\) is a superposition of the states that have energy \(E_n^{(0)}\). Let \(D_n\) be the set of indices \(m\) such that \(E_m^{(0)} = E_n^{(0)}\). Then we can write
\[\ksn{n} = \sum_{m \in D_n} c_m \ket{m}\]
for some coefficients \(c_m\). To find the \(c_m\), we turn to the Schr\"odinger equation. We have
\begin{align*}
\left(H_0 + \lambda H^\prime\right) \ket{\psi_n} &= E_n \ket{\psi_n} \\
\left(H_0 + \lambda H^\prime\right) \left[\ksnp{n}{0} + \lambda \ksnp{n}{1} + \Od{\lambda^2}\right]
&= \left(E_n^{(0)} + \lambda E_n^{(1)} + \Od{\lambda^2}\right) \left[\ksnp{n}{0} + \lambda \ksnp{n}{1} + \Od{\lambda^2}\right] \\
H_0 \ksnp{n}{0} + \lambda \left(H^\prime \ksnp{n}{0} + H_0 \ksnp{n}{1}\right) + \Od{\lambda^2}
&= E_n^{(0)} \ksnp{n}{0} + \lambda \left(E_n^{(0)} \ksnp{n}{1} + E_n^{(1)} \ksnp{n}{0}\right) + \Od{\lambda^2}.
\end{align*}
In the zeroth order of \(\lambda\) this gives \(H_0 \ksnp{n}{0} = E_n^{(0)} \ksnp{n}{0}\), confirming what we already know: \(\ksnp{n}{0}\) is an eigenstate of \(H_0\) with eigenvalue \(E_n^{(0)}\). In the first order of \(\lambda\) we get
\[H^\prime \ksnp{n}{0} + H_0 \ksnp{n}{1} = E_n^{(0)} \ksnp{n}{1} + E_n^{(1)} \ksnp{n}{0}.\]
Multiplying with \(\bra{m}\), where \(m \in D_n\),
\begin{align*}
\bra{m} H^\prime \ksnp{n}{0} + \bra{m}H_0\ket{\psi_n^{(1)}} &= {E_n^{(0)} \braket{m}{\psi_n^{(1)}}} + E_n^{(1)} \braket{m}{\psi_n^{(0)}} \\
\bra{m} H^\prime \ksnp{n}{0} + \cancel{E_m^{(0)}\braket{m}{\psi_n^{(1)}}} &= \cancel{E_n^{(0)} \braket{m}{\psi_n^{(1)}}} + E_n^{(1)} \braket{m}{\psi_n^{(0)}} \\
\braket{m}{\psi_n^{(0)}} &= \frac{\bra{m} H^\prime \ksnp{n}{0}}{E_n^{(1)}}.
\end{align*}



\begin{gap}
This part needs to be added. It should include discussion of the orthonormality of the states.
\end{gap}

\subsubsection{Rearranging the Schr\"odinger equation}

To get the first- and second-order corrections, we start by multiplying the Schr\"odinger equation by \(\bra{\psi_m^{(0)}}\). We get
\begin{align}
\bra{\psi_m^{(0)}} (H_0 + \lambda H^\prime) \ksn{n} &= E_n \braket{\psi_m^{(0)}}{\psi_n} \nonumber\\
E_m^{(0)} \braket{\psi_m^{(0)}}{\psi_n} + \bra{\psi_m^{(0)}} \lambda H^\prime \ksn{n} &= E_n \braket{\psi_m^{(0)}}{\psi_{n}} \nonumber\\
\braket{\psi_m^{(0)}}{\psi_n} &= \frac{\bra{\psi_m^{(0)}}\lambda H^\prime \ket{\psi_n}}{E_n - E_m^{(0)}} \nonumber\\
\braket{\psi_m^{(0)}}{\psi_n} &= \frac{\lambda\bra{\psi_m^{(0)}} H^\prime \left(\ksnp{n}{0} + \lambda \ksnp{n}{1} + \Od{\lambda^2}\right)}{E_n - E_m^{(0)}} \label{eq:tdpt.sefrac}
\end{align}
Let's examine the denominator. We'll consider the two cases where \(n\) is/is not degenerate with \(m\).

Let \(D_n\) be the set of indices \(m\) such that \(E_m^{(0)} = E_n^{(0)}\). If \(m \not\in D_n\) then
\begin{align*}
E_n - E_m^{(0)} &= {\left(E_n^{(0)} - E_m^{(0)}\right)} + \lambda E_n^{(1)} + \lambda^2 E_n^{(1)} + \Od{\lambda^3}\\
&= {\left(E_n^{(0)} - E_m^{(0)}\right)}\left[1 + \Od{\lambda}\right].
\end{align*}
Using the expansion \(\frac{1}{1+x} = 1 - x + x^2 - x^3 + \cdots\) with \(x = \Od{\lambda}\),
\[E_n - E_m^{(0)} = \left(E_n^{(0)} - E_m^{(0)}\right) \frac{1}{1 - \Od{\lambda}}\]
and plugging this in to (\ref{eq:tdpt.sefrac}) we get
\begin{align}
\braket{\psi_m^{(0)}}{\psi_n} &= \frac{\lambda \bra{\psi_m^{(0)}} H^\prime \left(\ksnp{n}{0} + \lambda \ksnp{n}{1} + \Od{\lambda^2}\right)}{E_n^{(0)} - E_m^{(0)}}\left[1+\Od{\lambda^2}\right] \nonumber\\
&= \frac{\lambda \bra{\psi_m^{(0)}} H^\prime \ksnp{n}{0}}{E_n^{(0)} - E_m^{(0)}}+ \Od{\lambda^2},
&& n \not\in D_n. \label{eq:tdpt.nDn}
\end{align}

If instead \(m \in D_n\) then
\begin{align*}
E_n - E_m^{(0)} &= \cancel{\left(E_n^{(0)} - E_m^{(0)}\right)} + \lambda E_n^{(1)} + \lambda^2 E_n^{(1)} + \Od{\lambda^3}\nonumber\\
&= \lambda E_n^{(1)} \left(1 + \lambda \left[\frac{E_n^{(2)}}{E_n^{(1)}} + \Od{\lambda} \right]\right)
\end{align*}
Again, we use the expansion \(\frac{1}{1+x} = 1 - x + x^2 - x^3 + \cdots\), but now with \(x = \lambda \left[\frac{E_n^{(2)}}{E_n^{(1)}} + \Od{\lambda}\right]\). We get
\[
E_n - E_m^{(0)} = \left(E_n^{(0)} - E_m^{(0)}\right) + \frac{\lambda E_n^{(1)}}{1 - \lambda \frac{E_n^{(2)}}{E_n^{(1)}} + \Od{\lambda^2}}.
\]
Plugging this in to (\ref{eq:tdpt.sefrac}) we get
\begin{align}
\braket{\psi_m^{(0)}}{\psi_n} &= \frac{\bra{\psi_m^{(0)}} H^\prime \left(\ksnp{n}{0} + \lambda \ksnp{n}{1} + \Od{\lambda^2}\right)}{E_n^{(1)}}\left[1- \lambda \frac{E_n^{(2)}}{E_n^{(1)}} + \Od{\lambda^2} \right] \nonumber\\
&= \frac{\bra{\psi_m^{(0)}} H^\prime \ket{\psi_n^{(0)}}}{E_n^{(1)}} + \lambda \left[\frac{\bra{\psi_m^{(0)}} H^\prime \ket{\psi_n^{(1)}}}{E_n^{(1)}} - \frac{E_n^{(2)} \bra{\psi_m^{(0)}} H^\prime \ket{\psi_n^{(0)}}}{{\left[E_n^{(1)}\right]}^2}\right] + \Od{\lambda^2},
&& n \in D_n. \label{eq:tdpt.Dn}
\end{align}

Now, we can write an expression for complete state. We have \(\ket{\psi_n} = \sum_m \ket{\psi_m^{(0)}} \braket{\psi_m^{(0)}}{\psi_n}\) and we can read off the \(\braket{\psi_m^{(0)}}{\psi_n}\) from (\ref{eq:tdpt.nDn}) and (\ref{eq:tdpt.Dn}). Putting this all together,
\begin{align}
\ksnp{n}{0} + \lambda \ksnp{n}{1} + \Od{\lambda^2} =
&\sum_{m \in D_n}\ket{\psi_m^{(0)}}\frac{\bra{\psi_m^{(0)}} H^\prime \ket{\psi_n^{(0)}}}{E_n^{(1)}}
\\\nonumber&+\lambda \left\{
\sum_{m \in D_n}\ket{\psi_m^{(0)}} \left[\frac{\bra{\psi_m^{(0)}} H^\prime \ket{\psi_n^{(1)}}}{E_n^{(1)}} - \frac{E_n^{(2)} \bra{\psi_m^{(0)}} H^\prime \ket{\psi_n^{(0)}}}{{\left[E_n^{(1)}\right]}^2}\right]\right.
\\\nonumber&\hspace{3em}\left.+ \sum_{m \not\in D_n} \ket{\psi_m^{(0)}}\frac{\bra{\psi_m^{(0)}} H^\prime \ksnp{n}{0}}{E_n^{(0)} - E_m^{(0)}}\right\}
\\\nonumber&+\Od{\lambda^2}. 
\end{align}
Equating orders of \(\lambda\), we have a zero-order equation
\begin{equation}\label{eq:tdpt.zeroordereq}
\boxed{\ksnp{n}{0} = \sum_{m \in D_n}\ket{\psi_m^{(0)}}\frac{\bra{\psi_m^{(0)}} H^\prime \ket{\psi_n^{(0)}}}{E_n^{(1)}}}
\end{equation}
and a first-order equation
\begin{equation}\label{eq:tdpt.firstordereq}
\boxed{\ksnp{n}{1} = \sum_{m \in D_n}\ket{\psi_m^{(0)}}
\left[\frac{\bra{\psi_m^{(0)}} H^\prime \ket{\psi_n^{(1)}}}{E_n^{(1)}} - 
\frac{E_n^{(2)} \bra{\psi_m^{(0)}} H^\prime \ket{\psi_n^{(0)}}}{{\left[E_n^{(1)}\right]}^2}\right]
+ \sum_{m \not\in D_n} \ket{\psi_m^{(0)}}\frac{\bra{\psi_m^{(0)}} H^\prime \ksnp{n}{0}}{E_n^{(0)} - E_m^{(0)}}.}
\end{equation}

\subsubsection{First order energy}

Multiplying (\ref{eq:tdpt.zeroordereq}) by \(\bra{\psi_n^{(0)}}\),
\begin{align*}
\braket{\psi_n^{(0)}}{\psi_n^{(0)}} &= \sum_{m \in D_n}\braket{\psi_n^{(0)}}{\psi_m^{(0)}}\frac{\bra{\psi_m^{(0)}} H^\prime \ket{\psi_n^{(0)}}}{E_n^{(1)}} \\
1 &= \sum_{m \in D_n}\delta_{mn}\frac{\bra{\psi_m^{(0)}} H^\prime \ket{\psi_n^{(0)}}}{E_n^{(1)}}
\end{align*}
so
\[
\boxed{E_n^{(1)} = \bra{\psi_n^{(0)}} H^\prime \ket{\psi_n^{(0)}}}.
\]
In words, the first order energy correction is the expectation value of \(H^\prime\) for the zero-order state.


\section{Identical particles}

Suppose we have a system of two identical particles with the wavefunction \(\Psi(x_1, x_2)\) where \(x_1\) and \(x_2\) are coordinates of the particles (taking into account things like position and spin).

\begin{fact}
Particles either have wavefunctions that are symmetric, with \(\Psi(x_1, x_2) = \Psi(x_2, x_1)\), or antisymmetric, with \(\Psi(x_1, x_2) = -\Psi(x_2, x_1)\).
\end{fact}

We call the symmetric type of particles bosons and the antisymmetric type of particles fermions.

\subsection{Pauli exclusion principle}

Consider a wavefunction \(\Psi(x_1, x_2)\) for two fermions. If \(x_1 = x_2 = x\) then we have \(\Psi(x, x) = - \Psi(x, x)\) so it must be that \(\Psi(x, x) = 0\). This shows that two identical fermions cannot occupy the same state.


\end{document}